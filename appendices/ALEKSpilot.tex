% arara: pdflatex: {files: [MathSACpr2014]}
% !arara: indent: {overwrite: yes}
\chapter{ALEKS pilot}\label{app:sec:aleks}
\section[MTH 20]{MTH 20 Several classes during 2012--2013 AY (Edwards)}

The pilot includes the extensive use of ALEKS, a technology based assessment learning system,  in 2 on campus and 2 online classes each term.  

Course logistics: 
\begin{itemize}
    \item Students are walked through an introduction to the system and given
      an assessment.
    \item Students are then provided with a very clear visual pie chart showing
      them what they know.
    \item ALEKS then provides students the opportunity to work on a range of
      instructor chosen topics at their current level.  Student only work on concepts they have not mastered.
    \item Explanations and videos are provided with each topic.
    \item Students are provided instant feedback and instant online teaching.
    \item Students are not given the option to skip work that they have not
      mastered, essentially forcing them to learn the material and 
	fill in the concepts gaps that they began the class with.
    \item Students are routinely assessed with new topics available as they
      move through the course.
    \item Students are in the computer lab working on ALEKS throughout the
      class period.
    \item Students (generally for whom the material is recent) have the
      ability to move ahead.  
\end{itemize}

\subsection{Results and Statistics}
I'm reflecting only Fall term MTH 20 students; this was a definite pilot.  
A variety of changes were incorporated into Winter and Spring terms 
which included additional lectures and assignments that had each class 
more closely resemble more traditional class.
\begin{itemize}
	\item Students loved the instant feedback.
    \item Students enjoyed the ability to work in the ALEKS system, choosing
      their topics, and getting ahead when desired.  There were very, very few
      complaints about the system.
    \item Students became aware of how much time they studied, with a clear
      visual of the relationship between study time and learning. 
    \item FOUR students last term completed the MTH 20 material, moved on to
      MTH 60 material, took and passed my MTH 60 final exam.
\end{itemize}

On Campus Classes:  
\begin{itemize}
    \item 78\% of students passed MTH 20 last Fall compared to 89\% using
      ALEKS  (7 {\sc a.m.}  class result was 63\% passed using ALEKS).
    \item Of those that went on to MTH 60:  60\% passed last Fall compared to
      69\% using ALEKS (7 {\sc a.m.} class result: 13\% passed, 1 in 8).
\end{itemize}
DL Classes:
\begin{itemize}
    \item 62\% of students passed MTH 20 last Fall compared to 71\% using
      ALEKS.
    \item Of those that went on to MTH 60:  61\% passed last Fall compared to
      46\% using ALEKS.  
\end{itemize}

\section[MTH 112]{Pilot in MTH 112 during Winter 2013 (Louie)}
The most beneficial aspect of ALEKS was the instant feedback and the chance for students to fill in the holes of their prerequisite knowledge. Regretfully, I can only provide data from a rather small survey; I compared one class (no ALEKS) to two classes (with ALEKS). The data from the ALEKS classes were averaged and compared with non ALEKS class; here is a summary of my findings:
\begin{itemize}
	\item there was no distinction for either grade distribution or overall pass rate between classes;
	\item in all three classes the pass rate was 73\% which is well above the current 57\% campus average pass rate;
	\item the attrition rate for non-ALEKS classes was 32\%. The ALEKS class averaged a mere 14.7\%;
\item 	when students were asked how they agreed with the statement, ``ALEKS helped me learn the concepts in this course'',  82.2\% responded that they agreed or strongly agreed;
\item	when students were asked how they agreed with the statement, ``ALEKS helped me learn concepts from previous math courses'', 75\% of the students responded that they agreed or strongly agreed.
\end{itemize}
The data may suggest that ALEKS has the potential to keep students working towards a goal and less likely to withdraw from the course. Another benefit is the ability to track time spent on required homework. The maximum (average) time spent on ALEKS was 15.4 hours and the minimum 1.6 hours per week. It was beneficial to gauge the amount of work students completed outside of class. Students could only earn full credit if they completed all of the homework which forced them to keep up with the course material. Lectures seemed to flow with little interruption. 

Despite my lack of data to support higher grades, I feel that ALEKS was beneficial to the students. I had originally planned to track how students performed in the next class, \nth{1} term Calculus. However, I found little difference in the pass rate of ALEKS students than those of non-ALEKS students. 

While my sample may be too small to draw statistical significance the effect that ALEKS had on student success, I feel that the overall outcome was positive. Student feedback suggested that they enjoyed using the program and felt that they were able to learn the concepts taught in the class. Lastly, the lower attrition rate may suggest students stayed involved in the class longer than those that did not use ALEKS. 