% arara: pdflatex: {files: [../MathSACpr2014]}
% !arara: indent: {overwrite: yes}
\chapter{Professional development highlights}\label{app:sec:professionaldevelop}

%\begin{widepage}
\begin{description}[style=nextline]
	\item[Farshad Barman (Full-time Instructor, Rock Creek Campus)]
	I was awarded a professional leave in the fall of 2012 and worked on my
	projects in astronomy for my students. I provided these projects to my
	colleagues for use in their classrooms. I gave a presentation to a group of
	young Hispanic students (Hermanas) sponsored by Intel in the winter of 2013 as
	a result of this professional leave.

	I gave a presentation on the history of math in astronomy to the ORMATY/WAMATYC
	math conference in 2011.
	\item[Jessica Bernards (Full-time Instructor, Rock Creek Campus)]
	I was chosen to be a part of AMATYC's Project ACCCESS program.  Within
	this cohort of 22 instructors from across the nation I learned a tremendous
	amount professionally in terms of best practices and accessibility.  Part
	of this program was to create a project to aid in student retention.  I
	created a study skills program to be used in our developmental math
	courses.  Through the format of videos, worksheets, and short class
	discussions, loss of vital class time is minimal with the program so that
	instructors can still have enough time to teach the curriculum.  The topics
	include: how learning math is different, how to do math homework, time
	management, effective listening and note taking, math and test anxiety, how
	to take math tests, and resources available for help.  This program is now
	being used nationwide in developmental math classes including Arizona,
	California, Washington, Philadelphia, Tennessee, and North Carolina
	community colleges.

	I am a Co-teacher of PCCs `One Session Classes to Improve your
	Instructional Skills' with Gabe Hunter-Bernstein. In this class we teach
	teachers how to apply principles of direct instruction to increase
	effectiveness of content delivery, apply principles of cooperative learning
	to improve student learning, and balance teacher- and student-centered
	learning approaches to appropriately meet learner needs.  By teaching this
	class I have not only made connections with faculty from various
	disciplines and colleges, but I've strengthened my own instruction as well.
	\item[dMarie Carver (Full-time Instructor, Sylvania Campus)]
	I attended the National Association for Developmental Education Conference
	(Feb 2012) with fellow DE instructors.  Upon return, I gave workshops to
	the full DE SAC on teaching/learning strategies acquired at the sessions I
	attended.  I continue to apply the strategies in my classes.

	I attended the Whitewater Institute Watermarks institute (Summer 2012) with
	Cecelia Guinee, a fellow DE Reading/Writing instructor.  This weeklong
	training session focused on collaboration, curriculum, and assessment.
	Cecelia and I worked together to create lessons on graphing for the Reading
	90 instructors.
	\item[Ann Cary (Full-time Faculty, Sylvania Campus)]
	I attended the Rogue Community College Summer Math Institute.  The event
	was a four-day workshop that explored the curriculum development and
	implementation of the career technical math course created at RCC as an
	alternative to MTH 60. I incorporated some of the mathematical models from
	this course into my pre-college courses and was pleased by the excitement
	students showed for them. I also used the knowledge that I gained at this
	workshop while participating on the NSF-IUSE grant proposal team.

	I gave an ORMATYC conference presentation.  I made improvements to my
	online course using techniques for enhancing student engagement that I was
	exposed to at the Infinite Possibilities Conference and the Creating
	Balance in an Unjust World Conference. I presented this work at the 2013
	ORMATYC conference. This presentation also included summaries from the
	article, Instructor-Student and Student-Student Rapport in the
	Classroom, and had participants interactively analyze effective and
	ineffective online communication.
\item[Daniel Castleton (Temporary Full-time Instructor, Cascade Campus)]

	No professional development reported.

	\item[Irina Didenko (Full-time Instructor, PCC Prep, Southeast Campus)]
	I attended the IECP (Institute on Early College Pedagogy) in 2008, the Peer
	Learning Conferences (PLC) for GTC Programs (2009, 2011, and 2012.) At these
	conferences, I learned of current trends, researches studies, practices in
	pre-college and college level mathematics courses. The conferences have
	educated me about models for integration, remediation and support, how to
	create dynamic  learning environments, support student success and persistence.
	I met with various math faculties to discuss more socially relevant topics into
	our classes such as process you can facilitate with your peers to: foster
	strong partnerships, develop cohesive teams, build cultural competency,
	increase program visibility and enrollments, understand and influence public
	policies pursue funding opportunities, etc. and using it in an educational
	environment to teach mathematics. I have used some of these strategies in
	courses to try to improve student understanding and success. I am incorporating
	this knowledge, experience, and understanding by creating my own Class
	Activity worksheets (guided notes) with built in activities for students to
	do in class and modifying other's materials that I plan to implement in my
	math courses.

	I presented at the Conference in 2012. I made a workshop `Math Curriculum
	Alignment for Gateway to College Program' Topics dealt with relevancy and
	learning, instructional focuses, accountability, community and engagement, and
	school structure.

	\item[Diane Edwards (Full-Time Instructor, Rock Creek Campus)]
	I gave a PCC Puentes Program design and Conference presentation.  I developed
	necessary material to teach and participate in the Fall Puentes (Bridge)
	program. The two week program was offered to PCC bound migrant students. I was
	involved in teaching students both math and strategies for succeeding in
	college math classes. I also presented results with colleagues at the 2013
	Student Success and Retention Conference.

	I attended the 2011 Anderson Conference on Retention. Among other things, I
	learned that students surveyed listed not knowing how study as an
	obstacle for success; 50\% of students study less than 10 hours weekly for ALL
	of their classes; More than 50\% of Rock Creek Students take DE math classes. As
	a result I formed MTH 20 study sessions, available to all RC MTH 20 students.
	I also go to most MTH 20 classes during the first two weeks of the term to
	introduce myself and invite and encourage students to participate.

	\item[Lisa Folberg (Full-time Instructor, Rock Creek Campus)]
	I have been the ORMATYC treasurer since 2007.  This position has allowed me to
	work more closely with colleagues from all over the state.  It has also allowed
	me to deal more directly with textbook reps than I otherwise would, providing
	me the opportunity to become more familiar with services available for
	enhancing our students' learning experience.

	\item[Ross Folberg (Full-time Instructor, Sylvania Campus)]
	I have been working for the last 5 years on the development of an alternative
	pathway for CTE majors at PCC (tentatively known as MTH 80 and 85).  This has
	led us to my involvement in a TAACCCT grant in 2011 and a NSF-TUES grant in
	2012 in which I was a co-principal investigator.  In the process I have opened
	up dialogue between math and CTE instructors at PCC and instigated the
	embedding of mathematicians into CTE classes to see what math they use, how it
	is being used, and the language and context of the applied mathematics.  This
	collaboration has also led me to become the only non-CTE faculty on the CTE
	chairs committee.  This work has also involved a significant amount of research
	into other NSF funded projects and work being done relating to CTE and
	contextualized math being done throughout the nation.

	I was integral in the implementation of the AMP (now MTH 07, MTH 08) review
	courses in Fall 2010 including working with publishers to allow and set up the
	use of their software in the course at no charge to the student.

	\item[Will Freeman (Full-time Instructor, Sylvania Campus)]

        No professional development reported.

	\item[Wendy Fresh (Full-time Instructor, Rock Creek Campus)]
	I attended the 2013 AMATYC conference in Anaheim, CA.  Attending this
	conference opened my eyes to many different ways of teaching.  While at the
	conference I was inspired to bring back to my campus the idea of a
	flipped classroom along with using free technology in the classroom.  I
	hope to present these ideas myself at our state conference in 2014.

	I was the MathFest school liaison in both 2011 and 2012.  Each spring, Portland
	Community College offers a Math Contest for the high school and middle school
	students within its district.  Students placing in the top 12\% of the
	Individual Competitions are then invited to participate in the Oregon
	Invitational Mathematics Tournament held at a rotation of 4-year universities.
	My role in 2011 and 2012 was to compile and organize a list of high schools and
	middle schools to invite.  This involved identifying contacts at each school,
	reaching out to these contacts and then organizing registrations of over 200
	students.

	\item[Matthew Funk (Faculty Chair, Southeast Campus)]
	In Nov 2010, I had the opportunity to give a presentation at the AMATYC
	Conference in Las Vegas, Nevada.  It is the \nth{2} such presentation that I have
	given over the past 10 years (I also presented at the AMATYC Conference in San
	Diego, CA, on the same topic); I wanted to show people a way of teaching the
	concept of functions to students who are liberal arts majors.  Many of these
	students don't see any need for understanding functions for their respective
	majors; through the use of function puzzles, I showed that the concept of
	functions can arise in many liberal arts fields, such as history, religion,
	geography, and literature.

	In Nov 2013, I had the opportunity to attend the AMATYC Conference in Anaheim,
	CA.  I was particularly inspired by Joseph Agnich, a faculty member from NVCC.
	He gave a 50-minute presentation on how one should work with developmental
	education students, as far as teaching mathematics is concerned.  He is one of
	the only people who was able to verbalize my concerns about developmental
	mathematics; I could have listened to him for hours, for what he discussed was
	very valuable.  Many of his techniques I already use, but I learned some
	additional techniques, such as going the extra mile to catch students early
	(those who are struggling), and how to get them back on the right track.  I
	hope, in the near future, to be able to invite him to PCC, to give a talk to
	our faculty, since developmental education is at the forefront of much of our
	current work.

	\item[Frank Goulard (Part-time Instructor, Sylvania Campus)]
	I am one of 14 members on Oregon's new Higher Education Coordinating
	Commission (HECC), 2013--present.  The HECC is charged with the funding, budget
	formulations, achievement compacts, higher education efficiencies, and student
	affordability issues as they relate, and integrate, to the Governor's
	40-40-20 plan for Oregonians by the year 2025. This affects every Oregon
	institution of higher education, including PCC.

	I am the National Exhibits Chair for the American Mathematical Association of
	Two-Year Colleges (AMATYC), the national mathematics in community colleges
	organization, 2011--present; Secretary, 2005--present, and Exhibitor Liaison,
	1992--present, for the Oregon Mathematical Association of Two-Year Colleges
	(ORMATYC). I collaborate with a six-person national conference committee to
	produce a successful annual national conference, hosting 1200 attendees each
	November. I coordinate with a five-person state conference committee /
	Executive Board to produce a successful annual state mathematics conference,
	hosting 150 attendees each April. Creating a cohesive relationship with
	exhibitors on the national and state levels ensures a closer working
	relationship, which is beneficial for the PCC Math SAC and student needs.

	\item[Peter Haberman (Full-time Instructor, Sylvania Campus)]
	I have been chair of the Academic Policies and Standards standing committee of
	the Educational Advisory Council (EAC) since January 2011.  In this role, I am
	part of the EAC's Leadership Team and I meet regularly with the Deans of
	Instruction, the Vice President of Academic and Student Affairs, and the
	District President.  This experience has given me great insight, understanding,
	and awareness about how the college functions and I've been able to bring
	important information to the Math SAC.

	\item[Dave Hall (Full-time Instructor, Rock Creek Campus)]
	No professional development reported.
	\item[Shane Horner (Full-Time instructor, Cascade campus)]
	I attended the NAEPP assessment conference in 2012.  At this 3-day conference
	in St. Louis we looked at the NAEPP assessment and working to see how the
	assessment tool matches community college and college entrance expectations.
	Because of this conference I now have a better understanding of how high school
	standards match our community college standards.

	I attended the STEMtech conference in 2013.  At this 4-day conference in
	Atlanta we attended an extraordinary number of sessions related to math,
	teaching, and the STEM field.  In addition to these topics, the national
	completion agenda was a major part of all the conversations.  This conference
	helped me understand some national trends as our school looks at the
	developmental math sequences.

	\item[Chris Hughes (Full-time faculty, Cascade)]
	I have been heavily involved with Distance Learning over the time period of
	this review; as the online faculty for math, I was on a very small steering
	committee to help decide how and where we would migrate from Blackboard in
	2009/2010, together with an EAC Distance Learning task force to evaluate the
	process for taking over course shells.

	I conducted an accessibility study with Scot Leavitt, Karen Sorenson, and Kaela
	Parks (Fall 2012); we have presented the findings at various conferences. The
	study has greatly impacted not only my day-to-day instruction, but our (Scot
	and my) interactions with the SAC and administrators; one of the largest and
	most measurable outcomes was the support and acknowledgement for WeBWorK. I
	consider the development of the Pre-College WeBWorK library with Alex Jordan (leader)
	and Carl Yao during Summer 2013 to be an unsung achievement that serves
	students on a daily basis. Other highlights include the development of MTH 84 (Introduction to
	\LaTeX) with Alex Jordan.

	\item[Alex Jordan (Full-time Instructor, Sylvania Campus)]
	I presented at the ORMATYC Conference (2013).  Chris Hughes and I gave a
	presentation about a course we developed to teach the typesetting software
	\LaTeX.  The course (MTH 84) is described in \vref{other:sec:mth84}. Presenting
	this course promoted use of professional typesetting software and highlighted
	its impact for efficient creation of instructional materials. The presentation
	also triggered discussions about how to implement online courses that teach a
	specific software skill.

	I traveled to a ``code camp'' in Vancouver BC in June 2013 to work with WeBWorK
	developers on new features for WeBWorK (see \cpageref{other:sec:webwork}).  I returned
	from the camp with knowledge instrumental in maintaining a PCC WeBWorK server
	and new insight for future application developments, some of which we are
	actively pursuing this year.

	\item[Kenneth Kidoguchi (Full time Instructor, Sylvania Campus)]
	I was a member of the EET/ENGR Hiring Committee.  Being a member of this
	interdepartmental committee improved my understanding of how mathematics is
	being used in engineering classes.  This resulted in adjustments to my
	presentation of mathematics topics to emphasize their relevance to engineering
	applications.

	I participated in several Maplesoft Webinars.  Ideas presented in these
	webinars were used as a starting point to build and/or update code for my
	student mapleTools.

	\item[Jerry Kissick (Faculty Chair, Rock Creek Campus)]
	After attending several sessions at AMATYC and ORMATYC conferences on ways to
	improve student involvement in the learning process in the classroom, I have
	implemented the use of guided class notes with built in activities for students
	to do in class with instructor oversight in MTH 111 and 112. Students say they
	are very helpful, especially if they have to miss a class.

	\item[Kandace Kling (Full-time Instructor, Sylvania Campus)]
	I took an online workshop on accessibility through Quality Matters fall quarter
	2011. The work I did in this workshop paired with the discussions we had in our
	accessibility committee and the research Scot Leavitt and Chris Hughes did
	caused me to take a more active role in making sure my classes are accessible.
	One of the changes I made was adding captioning (using Camtasia) to the videos
	I had previously made for my online classes.

	I took an online workshop on WeBWorK coding through MAA June 2013. The
	information I learned in this workshop paired with training I received from
	Alex Jordan enabled me to spend summer 2013 coding and editing WeBWorK problems
	to add to our PCC WeBWorK library to support our launch of WeBWorK based at PCC
	January 2014.

	\item[Ross Kouzes (Full-time Instructor, Sylvania Campus)]
	I organized small groups of colleagues to act as a peer mentoring group at
	Sylvania for various courses. We met multiple times per week to discuss best
	practices, activities and depth of content. Doing so more impacted my
	instructional style in a positive way than any other activities I have engaged
	in. These groups have also informed and motivated the forming of more formal
	groups and the mentoring practices at Sylvania.

	I wrote an extensive informational document regarding employment in the Math
	Department at PCC particularly designed for part-time faculty members which has
	been disseminated to all campus math departments. In doing so, I researched as
	many aspects of the college as I could imagine. It has influenced how I mentor
	new faculty.

\item[Jeff Lacks (Temporary Full-time Instructor, Sylvania Campus)]
	I attended several sessions of Sara Williams' MTH 20 class at Mt. Hood
	Community College and had frequent discussions with her afterwards about
	approaches to teaching pre-algebra to adults.

	I have written a packet of group work problems for MTH 95 at PCC modeled after 
	our MTH 251 lab manual. I have been improving on this for the last 5 years and 
	shared it with other MTH 95 instructors.

	\item[Scot Leavitt (Full-time Instructor, Sylvania Campus)]
	I interned through the Dr. Susanne Christopher Leadership Internship program
	(2009--2010).  This internship allowed me to complete a college-wide project
	(leading the implementation of the Cultural Literacy requirement for the AAOT
	degree at PCC) and to gain a deeper understanding of the curriculum and degrees
	process at PCC.  I developed a greater understanding of the "big picture" at
	PCC, which influences the work I do for any committee on which I serve.

	In 2013 I presented with Chris Hughes at the eLearning 2013 conference
	in February, at the Oregon Association of Higher Education and Disability
	Spring 2013 conference in May, and for the Oregon Community College Distance
	Learning Association in June.  Chris and I presented our work on accessibility
	and mathematics, which is documented elsewhere in this program review.  In
	addition to presenting at the eLearning conference, the sessions I attended
	there had a great impact upon my individual teaching.  Beside becoming aware of
	the issues with online courses faced at other colleges and seeing what the
	future might hold for online courses, I left the conference with two pages of
	"to do" items for both my online and on-campus courses.  In particular, I have
	held off teaching online this academic year (2013/14) until I am able to
	implement enough of the changes I took from the eLearning conference in order
	to bring my online course up to a level with which I am now comfortable.

	\item[Ronda Lively (Faculty Chair, Sylvania Campus)]
	I served on the ROOTS program advisory board for two years.  The other members
	on this board come from advising, counseling, student disabilities, learning
	center, and more. One major component of each meeting was discussions on
	factors that impact student success (both positive and negative), followed by
	brainstorming sessions on ways to increase success.

	I served on the Honors advisory board for two years.  For three terms, I taught
	one regular college algebra section and one honors section. This greatly
	impacted my teaching as I was able to try different structures, method of
	delivery, and activities in the honors classes.

	\item[Tammy Louie (Full-time Instructor, Cascade Campus)]
	I gave TLC Presentations at Cascade and Sylvania Campuses.  Annual
	presentations for the TLC Navigating rough waters: a guide from part-time
	to full-time. I was prompted to repeat this session over multiple requests
	regarding demand for part-time mentoring in the application process. The talk
	is not discipline specific and attendance has grown to 20 people (average) per
	session.

	I initiated and organized the \nth{1} Week Lecture Series at Cascade
	Campus.  I created the \nth{1} Week Lecture Series in January 2012 and sessions are
	repeated the first Friday of every term (except summer). The sessions focus on
	the prerequisite topics our students often forget from previous math classes.
	These topics include signed numbers, fractions, linear equations, linear
	graphs, exponents, factoring, functions and simplifying techniques. Numerous
	faculty members (both part and full time) volunteer their time to help
	students. The program has averaged over 150 students per term over the last two
	years.

\item[Rita Luetkenhaus (Temporary Job-share Instructor, Sylvania Campus)]
	I presented at the Sylvania Inservice Breakout sessions with two other
	instructors in other disciplines at PCC.  The presentation was a discussion and
	presentation of what we have learned from teaching online which included what
	works and what doesn't work.

	I attended an isee systems conference where I learned more about system
	dynamics and using it in an educational environment to teach mathematics and
	gained knowledge about how to create models that students can manipulate to
	understand variables in an equation and how they affect a dynamic system.  I am
	furthering this knowledge by building my own models and modifying other's
	models which I plan to implement in my algebra courses using a program called
	STELLA and share with the department.

	\item[Michael Marciniak, (Faculty Chair, Cascade Campus)]
	With Holli Adams conceived and implemented the AMP (now MTH 07, MTH 08) review
	courses in Fall 2010.  Through taking these one-week review courses in basic
	mathematics and elementary algebra, scores of students have been able to place
	into a higher math class than the one in which they were initially placed by
	COMPASS.   For this work, Holli and I were nominated by the administration and
	subsequently received a 2011 Excellence Award from NISOD (The National
	Institute for Staff and Organizational Development).

	\item[Michele Marden (Full-time Instructor, Sylvania Campus)]
	From taking PCC's first assessment class in 2010 to becoming the Learning 
	Assessment Chair (2012 to present), my work with assessment has been the 
	most intense professional development of my career. I have had radical
	shifts in my views of teaching and grading that have changed how I relate to
	students and colleagues. The work has significantly developed my awareness and
	knowledge of accreditation requirements for PCC and national trends in higher
	education. I am increasingly motivated to guide organizational change that
	supports student success in a wide-variety of ways. One focus is promoting a
	culture of assessment for both improvement (feedback loops that lead to
	improvements in curriculum and teaching) and accountability (assurance of a
	quality education for students that complete our courses, certificates, and
	degrees).

	I have explored ways to connect Math and CTE programs.  My work with CTE
	includes the state-required investigation of math used by CTE programs,
	teaching specialized MTH 60/65 courses (geared for for Building Inspection,
	Interior Design, Architecture, and Drafting), collaboration for the NSF TUES
	grant proposal, and classroom observations to better understand the
	mathematical needs of the CTE program. This work has solidified my opinion that
	the traditional algebra track is not the appropriate mathematical focus for the
	vast majority of our students, in fact it is often a barrier to success.
	Nationally, and at PCC, we need to broaden our math courses and pathways to
	support the mathematical needs of varied educational goals our students have
	and to develop a math-literate populace for our increasingly data-driven world.

	\item[Henry Mesa (Faculty Chair, Rock Creek Campus)]
	I have been reading articles and talking with colleagues concerning math
	placement.  Heiko Spoddeck sent out an article `Core Principles for
	Transforming Remedial Education: a joint Statement'  and a particular fact
	caught my attention and as it provided some justification for some actions that
	I have been taking as department chair.  The article basically stated that the
	more remedial courses a student has to take in order to get into college
	courses, the higher the probability of failure.  Of course the flip side is if
	you put someone in a course that is not ready will they not just fail outright
	and thus the same outcome will result.   For my particular situation of placing
	students into the right courses, I look at past mathematical history, their
	perception of ability during that time period, current time available, rate at
	which a student can learn a particular mathematical subject, persevere.

	In the Fall of 2011 I taught a CG 111C for students that have math anxiety.  I
	used the required text for the course by Paul D. Nolting, Math Study
	Skills.  The textbook confirmed some beliefs that I had and I use the
	statements from the textbook when I advise students.   The class also allowed
	me to see students in a
	different light. I could see how vulnerable some students are when it comes to
	the subject.  I have used that experience, and understanding of where a student
	may be at, when I engage with my own students.

	\item[Emily Nelson (Full-time Instructor, Rock Creek Campus)]
	I attended the TOTOM (Teachers of Teachers of Mathematics) conferences (2008,
	2009 and 2011).  This conference has educated me on current trends and changes
	in the mathematics education.  It has given good insight as to how the Common
	Core State Standards being implemented in the K--12 system are different than
	past standards. This will influence both how I teach the Math For Elementary
	Teachers sequence as well as how our DE math classes will need to change.

	I participated in the Core to College Alignment project and Math Advisory
	Committee (2012/2013).  My involvement in this project included determining how
	the mathematics standards in the Common Core match or don't match the
	content in MTH 111 at PCC. Results were compared among several other
	postsecondary institutions in Oregon who had done the same task.  We did this
	in an effort to learn some things about how aligned we are at the college level
	and determine common ground about what it means for a student to be college ready”.

	\item[Kimberly Neuburger (Full-time Instructor, Sylvania Campus)]
	For the 2009--2010 school year I was a participant in the PCC Leadership
	academy. This experience taught me about how to be a Level 5 leader which I
	have been able to use as both a member and co-chair of committees within the
	Mathematics SAC.

	I attended workshops offered by the college on accessibility. As a result of
	these, I was given the skills to make all of my online documents and videos
	fully accessible for MTH 60, 65, and 95. I have shared these accessible
	documents with all other instructors who were using them in their online
	courses.

	\item[Scott Perry (Full-time Instructor, Sylvania Campus)]
	No professional development reported.

	\item[Jeff Pettit (Full-time Instructor, Rock Creek Campus)]
	I participated in a one-term Assessment Course at Rock Creek campus along with
	Jess Bernards and Ann Cary. The course explored assessment techniques beyond
	normative assessment and focused on the implementation of portfolio assessment
	as well as providing students with a menu of assessments from which to choose.
	This course and the discussions I had with classmates (also PCC Instructors)
	from other disciplines directly affected my perspective of assessment as a
	method of communication and a guide to improvement instead of a position of
	judgment. This course greatly impacted my assessment methods in all my classes,
	and I now provide an assessment menu in most all courses I instruct. In
	addition, it also helped guide my perspective during discussions with
	colleagues regarding applying portfolios as an assessment method.

	I attended an all-day Case for Acceleration, Oregon Community Colleges
	conference with Diane Edwards at Lane Community College which outlined the
	California Acceleration Project, and offered first-hand experience with
	Statway, used in most California community colleges to accelerate students
	through developmental math courses and past statistics. I shared much of what I
	learned at the conference during Math SAC meetings. The issue of acceleration
	introduced at the conference has impacted my perspective as a member of
	PCC's Completion Investment Council as well as my involvement with the State
	Developmental Education Task Force. I shared my experiences with Statway with
	the Statistics Committee as they discuss changes in the statistics pathway.

	\item[Dennis Reynolds (Full-time Instructor, Rock Creek Campus)]
	I have attended several of PCC's Continuing Education Courses offered by
	Gabe Hunter-Bernstein (2012--2013).  The weekend workshops have helped
	improve my overall content delivery and assessments to my students. Full-day
	workshops over `Designing Effective Assessments' and `Applying
	Research to Instruction', used knowledge of Bloom's Taxonomy of the
	Cognitive Domain to increase instructional effectiveness and apply brain and
	motivation research to improve classroom practice. More specifically,
	`Matching Philosophy and Practice: Walking the Talk' reviewed common
	teaching philosophies. I re-evaluated and articulated my teaching philosophy to
	more effectively communicate and implement that philosophy in the classroom.

	I have researched and evaluated several Open-Source math software packages.
	Researching, evaluating, and implementing various open-sourced, math-related
	software tools have enhanced the quality of content delivery and evaluation for
	students. More specifically, LaTeX has improved the quality of documents and
	handouts for students, and WeBWorK provides online assessments for student
	learning. Research to implement these and other tools (sage, GeoGebra) is
	continuous, open-ended, and evolving.

	\item[Rebecca Ross (Full-time Instructor, Southeast Campus)]
	I attended the Conference on Research in Undergraduate Mathematics Education
	(RUME) by the Mathematical Association of America in 2011 and 2012.  At these
	conferences, I learned of current research studies in pre-college and college
	level mathematics courses.  I have used some of these strategies in courses to
	try to improve student understanding and success.  One example is where I
	instituted a research finding from CUNY College in New York requiring all
	developmental math students that failed their midterm to complete extra
	practice problems in order to be allowed to take the final.  The goal was to
	help students learn that by putting in extra time to practice the math we are
	learning in class, they will improve their understanding, test scores and
	ultimately, successful completion of the course.  Results were mixed but
	promising.

	I also enjoy reading current research and journal to remain knowledgeable of
	current mathematics education standards and research practices.  Although not
	directly math related, I also read the book The College Fear Factor---How
	Students and Professors Misunderstand One Another.  One great thing about this
	book was that much of the research took place at the community college level.  I learned of some of the prevailing fears my students have as they begin their studies as well as their expectations.  Some were surprising, like community college students don'�t generally like directed group learning, they prefer the instructor to `�instruct'� them; they feel this is what they paid for, not to teach themselves.

	\item[Steve Simonds (Faculty Chair, Sylvania Campus)]
	I attended the Math Summit at Lane Community College in the fall of 2012.  As a
	direct result of that I formed a SAC subcommittee to explore the Common Core
	Standards and the Smarter Balanced assessment tool.  I also served on two
	state-wide committees that looked into those same issues.  As an indirect
	result of my attendance at the Math Summit, I formed a committee to look into
	our dual credit program.  Results of that initiative are documented elsewhere
	in the program review.

	I attended a two-day diversity workshop led by Dr. Leticia Nieto on the
	Sylvania Campus Fall Term 2013.  The workshop helped me better understand the
	importance of considering and respecting how life experience influences others
	perception (as well as my own). After the conference, I was given the
	opportunity to (belatedly) join a team who has created a proposal for a
	Sylvania `Listening Intervention Team for Equity'. Loosely, the mission
	of the team is to create a supportive resource for challenging cross-cultural
	interactions, including incidents experienced as discriminatory.

	\item[Virginia Somes (Full-time Instructor, Cascade Campus)]
	At the AMATYC conference in November 2011, I attended a session given by Cheryl 
	Ooten on math anxiety.  She wrote a book on math anxiety and math study skills and 
	gave an excellent presentation on these topics.  Upon
	returning to PCC, I restructured my two MTH 20 classes in Winter 2012 to
	incorporate short study skills assignments every week.  In May 2012, I gave a
	TLC presentation on what I had learned from Cheryl Ooten's AMATYC
	presentation and her book on managing math anxiety with study skills.  My own
	experience incorporating short study skills lessons into my MTH 20 classes was
	the impetus for proposing an official study skills component in the proposed
	5-credit MTH 20.

	I attended the annual ORMATYC conference in April 2013.  While talking with
	colleagues from around the state, I learned of several iPad apps (ShowMe and
	EduCreations) that allow the user to create and share video lessons online.  I
	have since used ShowMe to create video lessons and examples on factoring
	techniques and graphing techniques for my MTH 65 classes.  I have also used
	ShowMe to answer questions students send me through email.  These videos have
	been very well received by my students.

\item[Thomas Songer (Temporary Job-share Instructor, Sylvania Campus)]
	I am an ongoing member of PCC DE Math Curriculum Reform Committee.  I first
	joined this committee in February 2012, when the focus was on MTH 60/65/70;
	since then, it has expanded into the much bigger project of reforming PCC's
	entire DE Math lineup.  This work has led me to study current literature on the
	significant issues with math education in the country, and learn about the vast
	ongoing math reform research efforts being explored---especially the urgent
	need for creating shorter and better math pathways for community college
	students.

	I was an attendee for the 2013--2014 PCC New Faculty Orientation and Institute
	Series.  This series covers seven day-events over a five-month period,
	terminating in the 2014 Anderson Conference.  These events have given me much
	greater awareness and contact with other district-wide faculty and our larger
	PCC management structure; increased my involvement with our PCC instructional
	support resources; and enhanced my perception of PCC's important evolving
	service priorities to our students.

	\item[Heiko Spoddeck (Full-time Instructor, Math Coordinator, Sylvania Campus)]
	I attended Beyond Inclusion, Beyond Empowerment Training (2012).  This training
	was special for me because it offered a more comprehensive view on diversity
	issues and a big picture that helped me put all the pieces together that I had
	learned in other trainings. It has helped me in my teaching and in working with
	people in general in that it explained the always present power dynamics
	between people who have agency (privilege) and people who are targets
	(marginalized). The training gave me a framework which made it easier for me to
	be aware of and verbalize these dynamics and therefore to speak up when
	discrimination occurs.

	I participated in L.E.A.D. Academy (2010--11) (Leadership Excellence and
	Development).  In this year-long training, I learned about leading people by
	supporting them to become their best, helping them through change, and
	encouraging them to grow. I also learned to appreciate the many different ways
	to lead.

\item[Greta Swanson (Temporary Full-time Instructor, Sylvania Campus)]
	I attended the Peer Learning Conference in Philadelphia, Pennsylvania in 2011
	as part of the Gateway to College Faculty. The conference focuses on improving
	DE education for students age 16--20 in high school completion programs.
	One of the talks addressed the needs of transsexual students on our campus.
	Since this talk I incorporated a short questionnaire on the first day of class
	asking, among other questions, the preferred name and pronoun(s) of the
	student. This limits assumptions and makes for a safer environment for the
	student.

	I took part in Portland Public School District's Beyond Diversity training
	in 2012. This two-day workshop addressed the part that race plays in education.
	It made me more aware of my own white privilege and the struggles that some of
	my students face in and outside of the classroom. I continue to make sure
	examples given in class representative of a wide variety of scenarios relevant
	to the student body.

	\item[Phil Thurber (Full-time Instructor, Sylvania Campus)]

	No professional development reported.

	\item[Emiliano Vega (Full Time Instructor, Sylvania Camps)]
	I was a member of the Chemistry Hiring Committee. Being a member on this
	interdepartmental committee helped improve my ability to bring chemistry
	examples and knowledge into my developmental mathematics classes, given
	students a broader view of where chemistry and mathematics overlap.

	I attended the MSJE (Math and Social Justice) conference in 2012 and 2013. This
	conference has and will to continue to show me ways in which we can empower
	students and their communities using mathematics and statistics as evidence.
	Following attendance at this conference, Ann Cary and I began a small work
	group (the MSJE work group) on finding and bringing social justice topics into
	our mathematics classes and also presented about this work group at the PCC
	Anderson Conference in 2013.

	\item[Carly Vollet (Full-time Instructor, Cascade Campus)]
	I was trained as a Critical Friends Group Facilitator and ran a
	multidisciplinary Critical Friends Group (CFG)at Cascade that met approximately
	once per month. A CFG consists of six to ten educators who commit to working
	together on a consistent basis toward better student learning. Members gather
	for meetings, at which they establish and publicly state student learning
	goals, collaborate around productive teaching practices, examine curriculum and
	student work, and identify school-culture issues that affect student
	achievement. Critical Friendship provides a forum for professional development
	that focuses on developing collegial relationships and encouraging reflective
	practice in order to increase student achievement.  By examining student work
	and adult work through collaborative reflection, educators hold themselves
	accountable for continuous improvement toward helping every student learn.  I
	also co-facilitated a workshop with Virginia Somes using a Critical Friends Group Protocol at ORMATYC.

	\item[Jon Wherry (Full-time Instructor, Rock Creek Campus)]
	In the fall of 2012, I attended the New Faculty Institute. Through this
	program, I learned about several support systems and resources available to
	instructors. For example, I learned about the Teacher Learning Centers and took
	interest in the Critical Friend's Group. The latter group gave me exposure to
	multiple disciplines and helped to inform me of best practices.

	I attended the STEMtech conference in Atlanta, GA in the fall of 2013. This
	conference taught me about some of the technologies available for students as
	well as the issues of implementing STEM and technologies at the two year
	community college setting. The incorporation of STEM is important to me, and I
	have reached out to faculty from other departments to provide a richer
	curriculum.

	\item[Carl Yao (Full-Time Instructor, Southeast Campus)]
	In summer 2013 term, a group of PCC math teachers created a free online
	homework system for MTH 60 and MTH 65. Based on this system, I wrote lecture
	notes (some created by other PCC math instructors) and put together captioned
	video lectures, and created two free online courses, MTH 60 and MTH 65.  In my
	ALC self-paced math class, if a student is working on MTH 60 and MTH 65 content,
	he/she can simply use these free online courses, and there is no need to
	purchase a textbook (costing about \$110).  Actually, any PCC student can use
	these resources to go through MTH 60 and MTH 65 content, and then take PCC's
	placement test to try to test into a higher level math course.

	\item[Stephanie Yurasits (Full-Time Instructor, Southeast Campus)]
	I was accepted into a Project ACCESS faculty cohort.  Project ACCCESS is a
	mentoring and professional development initiative for two-year college
	mathematics faculty. The project's goal is to provide experiences that will
	help new faculty become more effective teachers and active members of the
	broader mathematical community.  Twenty-four applicants are chosen to work
	together over two years to help better their instruction in the classroom.  I
	am in the process of creating a project and plan to implement it in Spring
	2014.
\end{description}
%\end{widepage}