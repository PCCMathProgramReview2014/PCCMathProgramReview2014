% arara: pdflatex: {files: [../MathSACpr2014], options: "--output-directory=../"}
% !arara: indent: {overwrite: yes}
\chapter{Instructor Qualifications}\label{app:sec:instructorquals}
\subsection{Mathematics Instructor Qualifications (prior to May 2011)}
Master's degree (MA or MS) in mathematics from an accredited college or university.  Or, a graduate degree in a related field with successful completion of at least 30 quarter credits of graduate level mathematics courses.

There are three alternative approval paths for part-time Math instructors.
\begin{enumerate}
  \item PSU Mathematics Graduate Students with 27 or more Graduate-level Credits in Mathematics: Any math course appropriate for the graduate student, i.e., it's not limited to certain courses. Approved many years ago by then VP for Academic Services Jim Van Dyke, we're allowed to hire PSU mathematics grad students who have completed 27 or more credits of graduate-level math courses. This path was worked out as part of a cooperative program with PSU's Mathematics Department. It gives us access to instructors and gives PSU grad students an opportunity to see if teaching is a good career fit.
  \item For Teaching MTH 30 through MTH 95: Approved several years ago by the Math SAC, instructors may teach MTH 30 through MTH 95 provided they have the following credentials.
    \begin{itemize}
      \item Bachelor's degree in Mathematics or Mathematics Education or in Education with an emphasis in mathematics. (Note: bachelor's degree in Business may be substituted for instructors of MTH 30.)
        AND
      \item Three years full-time (or equivalent cumulative part-time) mathematics teaching experience within grades 7 through 16.
        AND
      \item Transcript showing successful completion of a full calculus sequence.
    \end{itemize}
  \item Master's degree in a related field such as, but not limited to, Physics or Engineering: Any math course appropriate for the instructor, i.e., it's not limited to certain courses. Approved this year by the Math SAC, individuals may demonstrate competency by having a master's degree in a related field including, but not limited to, Physics or Engineering. The rationale behind this approval path is that folks with master's degrees in Physics or Engineering, and other fields, have many math-intensive graduate-level courses in their discipline.
\end{enumerate}

\subsection{Mathematics Instructor Qualifications (approved May 2011)}
\subsubsection{MTH 99 AND BELOW}
Master's degree (MA or MS) in Mathematics or Mathematics Education from a regionally accredited college or university.  Or a graduate degree in a related field\footnote{The applicability of a particular degree as `a related field' will be determined by an appropriate Division Dean in consultation with a Mathematics Faculty Department Chair.
    }, such as, but not limited to, Physics or Engineering, with successful completion of at least 27 quarter credits of graduate level mathematics courses.
\subsubsection{MTH 100 AND ABOVE}
Master's degree (MA or MS) in Mathematics from a regionally accredited college or university.  Or a graduate degree in a related field (as above), such as, but not limited to, Education, Physics, or Engineering, with successful completion of at least 27 quarter credits of graduate level mathematics courses.
\subsubsection{Criteria for provisional instructors MTH 99 and below }
\begin{enumerate}
  \item Master's degree in a related field (as above) such as, but not limited to, Physics or Engineering; OR
  \item Mathematics graduate students who are actively working on a degree (taking at least 1 credit per year) and have successful completion of at least 27 quarter credits in graduate level Mathematics courses on their transcript; OR
  \item Bachelor's degree in Mathematics or Mathematics Education or in Education with an emphasis in mathematics.  (Note: bachelor's degree in Business may be substituted for instructors of MTH 30.)
    AND 
    \begin{itemize}
      \item Transcript showing successful completion of a full year of calculus.
        AND
      \item Three years full-time (or equivalent cumulative part-time) mathematics teaching experience within grades 6 or above.
    \end{itemize}
\end{enumerate}
\subsubsection{Criteria for provisional instructors MTH 100 and above }
\begin{enumerate}
  \item Master's degree in a related field (as above) such as, but not limited to, Physics or Engineering; OR
  \item Mathematics graduate students who are actively working on a degree (taking at least 1 credit per year) and have successful completion of at least 27 quarter credits in graduate level Mathematics courses on their transcript; OR
  \item Master's degree in Mathematics Education or Education may be substituted for instructors of MTH 211, MTH 212, and MTH 213.
\end{enumerate}
\subsection{Mathematics Instructor Qualifications (approved February 2013)}
\subsubsection{MTH 99 and below}
\begin{enumerate}
  \item Master's degree (MA or MS) in Mathematics or Mathematics Education from a regionally accredited college or university; OR
  \item A graduate degree in a related field (as above), such as, but not limited to, Physics or Engineering, with successful completion of at least 30 quarter credits of graduate level mathematics courses.
\end{enumerate}
\subsubsection{MTH 100 and above}
\begin{enumerate}
  \item  Master's degree (MA or MS) in Mathematics from a regionally accredited college or university; OR
  \item A graduate degree in a related field (as above), such as, but not limited to, Education, Physics, or Engineering, with successful completion of at least 30 quarter credits of graduate level mathematics courses.
\end{enumerate}
\subsubsection{Demonstrated competency MTH 99 and below}
\begin{enumerate}
  \item Master's degree in a related field (as above) such as, but not limited to, Physics or Engineering; OR
  \item Bachelor's degree in Mathematics or Mathematics Education or in Education with an emphasis in mathematics AND Transcript showing successful completion of a full year of calculus AND three years full-time (or equivalent cumulative part-time) mathematics teaching experience within grades 6 or above.  
  \item Bachelor's degree in Business may be substituted for instructors of MTH 30.
\end{enumerate}
\subsubsection{Demonstrated competency MTH 100 and above}
\begin{enumerate}
  \item Master's degree in a related field (as above) such as, but not limited to, Physics or Engineering; OR
  \item Master's degree in Mathematics Education or Education may be substituted for instructors of MTH 211, MTH 212, and MTH 213.
\end{enumerate}
\subsubsection{Provisional approval MTH 99 and below}
Mathematics graduate students who are actively working on a degree (taking at least 1 credit per year) and have successful completion of at least 27 quarter credits in graduate level Mathematics courses on their transcript.
\subsubsection{Provisional approval MTH 100 and above}
Mathematics graduate students who are actively working on a degree (taking at least 1 credit per year) and have successful completion of at least 27 quarter credits in graduate level Mathematics courses on their transcript.

