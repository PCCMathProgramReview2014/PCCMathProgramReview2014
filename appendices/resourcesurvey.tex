% arara: pdflatex: {files: [../MathSACpr2014], options: "--output-directory=../"}
% !arara: indent: {overwrite: yes}
\chapter{Resource survey results}\label{app:sec:resourcesurvey}
Survey data details: $n=976$ face to face and $n=291$ online responses.

\begin{enumerate}
	\item Have you used any of the resources available through the library (e.g. calculator, netbook, or iPad rentals, textbook checkouts, scanners, or online database search engines) during your time as a student in a PCC math course?
	\begin{enumerate}
		\item     Yes. I frequently used these resources.
		\item    Yes, but I seldom/rarely used these resources.
		\item     No, but I knew that such resources were available.
		\item     No, and I was unaware that such resources were available.
	\end{enumerate}
	We found that our students both in face-to-face and online classes are generally knowledgeable about library and out of the classroom resources such as calculator rentals, netbook and iPad rentals, textbook checkouts, scanners and online searchable databases.
		
    \begin{tabularx}{\linewidth}{lY}
		\toprule
		              & Students knowledge of library and out-of classroom resources \\
		\midrule
		Face-to-face  & $81.45\%$                                                   \\
		Online/hybrid & $74.25\%$                                                   \\
		\bottomrule
	\end{tabularx}
	
	Not surprising that library and other out-of-the-classroom information is being used more frequently by our face-to-face students than that of online students. This could be due to less frequent visits to campus for online students and/or online students already have the resources available to them. 
	
	\begin{tabularx}{\linewidth}{lY}
		\toprule
		              & Actual use of library and out-of-classroom resources \\
		\midrule
		Face-to-face  & $48.76\%$                                            \\
		Online/hybrid & $25.77\%$                                            \\
		\bottomrule
	\end{tabularx}
	\item Were the library and related resources listed on your most recent math course syllabus?
	\begin{enumerate}
		\item Yes, it is listed on the syllabus with links.
		\item Yes, it is mentioned but no links are provided.
		\item No, it is not listed as a resource.
		\item I don't have a copy of the syllabus available.
	\end{enumerate}
	We found that both Part-time faculty and Full-time faculty included information regarding library and out-of-classroom resources on their syllabi.
	
	\begin{tabularx}{\linewidth}{lY}
		\toprule
		                  & Percentage of classes where the syllabus included resources \\
		\midrule
		Part-time faculty & $69.41\%$                                                   \\
		Full-time faculty & $69.48\%$                                                   \\
		\bottomrule
	\end{tabularx}
	
	The data suggests that there was very little distinction of which classes encourage more students to use outside resources in both our college level and pre-college level mathematics. 
	
	\begin{tabularx}{\linewidth}{lY}
		\toprule
		                  & Percentage of classes where the syllabus included resources \\
		\midrule
		College level     & $70.05\%$                                                   \\
		Pre-College level & $68.83\%$                                                   \\
		\bottomrule
	\end{tabularx}
	\item Does your current math course have online homework and/or online assessments available (e.g. WeBWorK, MyStatLab, MyMathLab, ALEKS)?
	\begin{enumerate}
		\item Yes, it is required.
		\item Yes, but it is optional.
		\item No such resource is available.
	\end{enumerate}
	Online homework has grown in popularity over the past few years. There has been much debate within our SAC if students should be required to use online homework in face-to-face and online classes. The question has often been raised if students should be required to pay an extra cost for such features and if so, what is a reasonable cost to the student? The data shows a general trend that online homework programs such as WeBWorK,  MyMathLab, MyStatLab, and ALEKS are being used more frequently in online than face-to-face classes. 
	
	\begin{tabularx}{\linewidth}{lY}
		\toprule
		              & Percentage of classes requiring online homework \\
		\midrule
		Face-to-face  & $13.93\%$                                       \\
		Online/hybrid & $70.45\%$                                       \\
		\bottomrule
	\end{tabularx}
	
	Data suggests that significantly more Full-time instructors are offering some form of online homework (either required or optional) than that of Part-time instructors. This discrepancy may reflect the need to convey and distribute more information about these programs should Part-time instructors want to offer similar options to their students.
	
	\begin{tabularx}{\linewidth}{lY}
		\toprule
		                  & Percentage of classes offering some form of online homework \\
		\midrule
		Full-time faculty & $70.78\%$                                                   \\
		Part-time faculty & $54.93\%$                                                   \\
		\bottomrule
	\end{tabularx}
	\item I am willing to pay up to \$35 extra for access to online homework and resources that may help me succeed.
	\begin{enumerate}
		\item Strongly agree
		\item Agree
		\item Neutral
		\item Disagree
		\item Strongly disagree
	\end{enumerate}
	When asked if students would be willing to pay up to \$35 to access online homework and resources that may help them to succeed, we found that online students were more willing to pay an extra fee. It should be mentioned that we previously mentioned data that online students were more likely to have used online homework and hence be better equipped to compare cost versus benefit. In contrast, a student who has not been previously exposed to an online homework system may not be able to properly address possible benefits and instead answer purely based on willingness to pay the given dollar amount. 
	    
	\begin{tabularx}{\linewidth}{lY}
		\toprule
		              & Percentage of student willing to pay for online homework   \\
		\midrule
		Face-to-face  & $18.44\%$                                                  \\
		Online/hybrid & $42.61\%$                                                  \\
		\cmidrule{2-2}
		              & Percentage of student unwilling to pay for online homework \\
		\cmidrule{2-2}
		Face-to-face  & $56.86\%$                                                  \\
		Online/hybrid & $27.14\%$                                                  \\
		\bottomrule
	\end{tabularx}
	
	Note that the above values do not include the students who responded `neutral' on the question as these differences were not statistically significant. 
	\item What Learning Management Software are available for your math course? Bubble in all that apply.
	\begin{enumerate}
		\item  Instructor web page
		\item  D2L and/or MyPCC
		\item  MyMathLab or MyStatLab
		\item  Other
		\item  None of the above
	\end{enumerate}
	\item Of the available Learning Management Software, which ones have you used? Bubble in all that apply.
	\begin{enumerate}
		\item  Instructor web page
		\item   D2L and/or MyPCC
		\item   MyMathLab or MyStatLab
		\item   Other
		\item   None of the above
	\end{enumerate}
    We found that a majority of our courses are using outside resources to
    connect with students. These resources include but are not limited to
    personal instructor websites, D2L, MyPCC, MyMathLab, MyStatLab, etc. 
	
	\begin{tabularx}{\linewidth}{lY}
		\toprule
		              & Percentage of classes offering additional resources \\
		\midrule
		Face-to-face  & $89.75\%$                                           \\
		Online/hybrid & $99.31\%$                                           \\
		\bottomrule
	\end{tabularx}
	
	A larger separation existed for Part-time instructors who do not use any of the above mentioned resources. This could be due to lack of information or lack of knowledge about available resources.
	
	\begin{tabularx}{\linewidth}{lY}
		\toprule
		                  & Percentage of classes offering additional resources \\
		\midrule
		Full-time faculty & $95.32\%$                                           \\
		Part-time faculty & $86.72\%$                                           \\
		\bottomrule
	\end{tabularx}
	
	Overall MyMathLab and MyStatLab are used more frequently in pre-college level classes in contrast to college level classes. 
	
	\begin{tabularx}{\linewidth}{lY}
		\toprule
		                  & Percentage of classes offering MML or MSL \\
		\midrule
		College level     & $31.49\%$                                 \\
		Pre-College level & $48.54\%$                                 \\
		\bottomrule
	\end{tabularx}
	
	\item What resources available from the PCC Math Department have you used? Bubble in all that apply.
	\begin{enumerate}
		\item  Course supplements
		\item  Calculator manuals
		\item  MTH 251 Lab Manual
		\item  Other
		\item  None of the above
	\end{enumerate}
    Our math department website offers additional materials for students. This
    includes course specific supplements to the textbook, calculator manuals
    specific to PCC math courses, required Calculus I lab, and other
    information regarding course description. Students may print these
    materials for free from any PCC computer lab. 
	\item What graphing software programs have you used? Bubble in all that apply.
	\begin{enumerate}
		\item  WolframAlpha
		\item  Graph
		\item  WinPlot
		\item  Other (e.g. Fooplot, Maple, GeoGebra)
		\item  None of the above
	\end{enumerate}
	
	\begin{tabularx}{\linewidth}{lY}
		\toprule
		                                     & Resources used by students in College Level Courses \\
		\midrule
		Wolfram Alpha                        & $24.88\%$                                           \\
		Graph                                & $14.90\%$                                           \\
		Winplot                              & $6.14\%$                                           \\
		Other (Maple, GeoGebra, FooPlot, etc.) & $27.34\%$                                           \\
		None of the above                    & $51.77\%$                                           \\
		\bottomrule
	\end{tabularx}
  \item Which of the following resources available at PCC have you used? Bubble all that apply.
    \begin{enumerate}
      \item  On-campus Student Learning Centers
      \item  Online tutoring
      \item  The Student Help Desk
      \item  Other (e.g. Collaborate or Elluminate)
      \item  None of the above
    \end{enumerate}
    We encourage students to use some of the resources that PCC offers such as On-campus Student Learning centers, online tutoring, student help desk, Collaborate and/or Elluminate. We found that a significant amount of students in Face-to-Face classes were using the resources whereas students enrolled in an online class were not.  This is not especially surprising since the nature of online courses allows infrequent campus visits for the student. However, we could work to encourage the use of online tutoring to our online demographic.

	\begin{tabularx}{\linewidth}{lY}
		\toprule
		              & Percentage of students using PCC learning resources\\
		\midrule
		Face-to-face  & $67.32\%$                                                   \\
		Online/hybrid & $36.08\%$                                                   \\
		\bottomrule
	\end{tabularx}
  \item Which of the following resources do you use for your math class that is available outside of PCC?  Bubble all that apply.
    \begin{enumerate}
      \item Private Tutoring
      \item  Math websites  (such as Khan Academy, Purple Math, etc.)
      \item  YouTube videos not provided by instructor
      \item  Other
      \item  None of the above
    \end{enumerate}
With the wide-spread availability of the internet, students have been increasingly using sites like Khan academy, PatrickJMT, PurpleMath, YouTube etc.\ to supplement class time. In the absence of formal lecture, the data suggests online students using these services more than their face-to-face classmates. For others, private tutoring or help from their peers is another option. 

\begin{tabularx}{\linewidth}{lY}
		\toprule
		              & Percentage of students using external web videos like Khan, PatrickJMT, PurpleMath, etc. \ldots\\
		\midrule
		Face-to-face  & $45.49\%$                                                   \\
		Online/hybrid & $56.36\%$                                                   \\
		\bottomrule
	\end{tabularx}

    The data suggests that both Pre-college and College Level courses/instructors/students are using these
    resources. It isn't surprising to see these resources used more readily by
    College Level students based on word of mouth or more knowledge of which
    sites are reputable and which are not. The more math classes the student
    takes, the more resources they can use to assist in their learning. 

    \begin{tabularx}{\linewidth}{lY}
		\toprule
		              & Percentage of students using some form of learning resource outside of the PCC network. \\
		\midrule
		College level     & $79.57\%$                                 \\
		Online/hybrid & $63.47\%$                                 \\
		\bottomrule
	\end{tabularx}

\end{enumerate}

