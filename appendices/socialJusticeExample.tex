%=====================================================================================================

%       Exponential Function Introduction

%				Created 201201

%				Edited 201304

%				LaTeX=>PDF sizing, graphics
%=====================================================================================================
% arara: pdflatex

\documentclass[11pt]{amsart}
\usepackage{amssymb,latexsym}
\usepackage{graphicx}
\usepackage{amscd}
\usepackage{multicol}
\usepackage{multirow}
\usepackage{amsmath}
\usepackage{calc}
\usepackage{enumitem}
\usepackage{fancyhdr,lastpage}
\usepackage{setspace}
\usepackage{caption}
\usepackage{framed}
\usepackage{wasysym}
\theoremstyle{definition}
\pagestyle{fancy}
\newtheorem{theorem}{Theorem}
\newtheorem{lemma}{Lemma}
\newtheorem{defi}{Definition}
\newtheorem*{notation}{Notation}
\newtheorem{ex}{Example}
\newtheorem{proc}{Process}
\newtheorem{gw}{Group Work}

%These are manual page settings
\setlength{\textwidth}{7.25in}			%for =>PDF, use 6.5in
\setlength{\textheight}{9.5in}		%for =>PDF, use 9.5in
\setlength{\oddsidemargin}{-0.375in}		
\setlength{\evensidemargin}{-0.375in}		
\setlength{\voffset}{-.75in}			%for=>PDF, use -0.75in
\setlength{\footskip}{15pt}



\usepackage{placeins} % enables \FloatBarrier, useful for positioning figures/tables more precisely.


\usepackage{pgfplots}

\tikzset{>=stealth}
\usetikzlibrary{backgrounds,arrows}
\tikzset{tight background}


\pgfplotsset{every axis/.append style={width=7cm,height=7cm,
					axis x line=middle,
					axis y line=middle,
					line width=0.75pt,
					tick label style={font=\tiny},
					label style={font=\small},
					legend style={font=\tiny}
}}

\pgfplotsset{framed/.style={axis background/.style ={draw=black!75}}}


\usepackage{hyperref}
%\usepackage{url}
\hypersetup{colorlinks=true,urlcolor=blue,linkcolor=blue,breaklinks=true}

\usepackage{xcolor}
%		My colors
\definecolor{silver}{rgb}{0.95,0.95,0.95}
\definecolor{mypurple}{cmyk}{0.5,1,0,0}			%purple
\definecolor{myblue}{cmyk}{1,0.012,0,0}		%blue
\definecolor{mygreen}{cmyk}{0.3,0,0.2,0.16}		%green
\definecolor{myred}{cmyk}{0,1,1,0}


\setlength{\parindent}{0pt}


\renewcommand{\labelenumi}{ (\alph{enumi}) }			% makes enumeration (a) instead of (a).



\begin{document}

\pagestyle{plain}

\onehalfspace

%============================================================
%           Header Stuff   CHANGE FOR EACH SECTION!!!
%============================================================

\fancyhf{}   % clears both header and footer
\fancyfoot[RE,RO]{ \scriptsize{Page \thepage\ of \pageref{LastPage}}}
\fancyfoot[LE,LO]{\scriptsize{Instructor:  A.E.Cary}}
\fancyhead[RE,RO]{\scriptsize{Section 4.3: Exponential Functions}}
\fancyhead[LE,LO]{\scriptsize{MTH 111 Lecture Notes}}
\renewcommand{\headrulewidth}{0.4pt} % Removes header line if 0pt
\fancyfootoffset[LE,LO]{0in}        %Moves center ??
\renewcommand{\footrulewidth}{0.4pt} % Removes header line if 0pt



%============================================================
%           Title/ Info
%============================================================

\begin{center}
  \larger[3] {MTH 111 Lecture Notes} \\
\end{center}

\section*{\larger[2]  Section 4.3: Exponential Functions}

\vfill

In 1988, a judge in Yonkers, New York instituted an \emph{exponential} fine on the city of Yonkers. Below is the background and scenario, published in the New York Times\footnote{\url{http://www.nytimes.com/1988/09/10/nyregion/yonkers-legal-battle-how-it-unfolded.html}}:

\begin{framed}
{\fontfamily{pcr}\selectfont \small{
		\textbf{Dec.\,1, 1980}: Justice Department sues Board of Education, City of Yonkers and Yonkers Community Development Agency, charging that the city racially discriminated in education and public housing.

\textbf{Nov.\,20, 1985}: Judge Leonard B. Sand of Federal District Court in Manhattan rules that Yonkers's housing and schools were intentionally segregated by race. A housing remedy order directs the city to build 200 units of public housing and to plan additional subsidized housing.

\textbf{Jan.\,28, 1988}: City Council approves consent decree that sets timetable for building 200 units of public housing and commits city to an additional 800 subsidized units.

\textbf{July 26, 1988}: Court sets Aug.\,1 deadline for Council to adopt zoning amendment needed to build the 800 units.

\textbf{Aug.\,1, 1988}: Council rejects amendment in a 4-to-3 vote.

\textbf{Aug.\,2, 1988}: Judge Sand finds city and the four Councilmen who voted against the amendment in contempt of court and imposes fines. The city's fines start at \$100 and double every day. The Councilmen's fines start at \$500 a day and increase by \$500 each day.
}}
\end{framed}

\vspace{11pt}

\begin{ex}
 Let $P$ be the amount fined (in dollars) $t$ days after the fines were imposed. Complete the entries in Table~\ref{tab:cm} and Table~\ref{tab:cy}.

\end{ex}
\renewcommand{\arraystretch}{1.5}

\begin{minipage}{0.5\linewidth}
	\begin{center}
		        \captionof{table}{Councilmen}
		        \label{tab:cm}
        \begin{tabular}{| p{1cm}|p{2cm}|p{4cm}|}
          \hline
          		$t$	&  $P$ 	&	Formula\\ \hline
            $0$			& 								&			\\ \hline
            $1$ 		& 								&			\\ \hline
            $2$ 		& 								&			\\ \hline
          	$3$ 		&    							&			\\ \hline
          	$4$ 		&       					&			\\ \hline
          	$5$ 		&   							&			\\ \hline
          	\vdots	&				\vdots		&	\vdots	\\[-11pt] \hline
          	$t$ 		&									&			\\ 
          	\hline
        \end{tabular}
	\end{center}
\end{minipage}
\begin{minipage}{0.5\linewidth}
	\begin{center}
		        \captionof{table}{City of Yonkers}
		        \label{tab:cy}
        \begin{tabular}{| p{1cm}|p{2cm}|p{4cm}|}
          \hline
          		$t$	&  $P$ 	&	Formula\\ \hline
            $0$			& 								&			\\ \hline
            $1$ 		& 								&			\\ \hline
            $2$ 		& 								&			\\ \hline
          	$3$ 		&    							&			\\ \hline
          	$4$ 		&       					&			\\ \hline
          	$5$ 		&   							&			\\ \hline
          	\vdots	&				\vdots		&	\vdots	\\[-11pt] \hline
          	$t$ 		&									&			\\
          	\hline
        \end{tabular}
	\end{center}
\end{minipage}

\newpage

\begin{ex} Graph each of the functions you found that model the fines for the Councilmen and the city of Yonkers. Identify the key features of each graph.

\begin{minipage}{0.5\linewidth}
\begin{center}
	\captionof{figure}{Councilmen}
	\label{fig:cm}
	   	\begin{tikzpicture}

			\begin{axis}[
			height=2in,
			width=2.75in,
			grid=major,
				xmin=0,xmax=6,
				ymin=0,ymax=3600,
				xlabel={$t$},
				ylabel={$P$},
				xlabel style={at={(ticklabel cs:0.5)},anchor=near ticklabel},
      	ylabel style={at={(ticklabel cs:0.5)},rotate=90,anchor=near ticklabel},
				xlabel={$t$, time in days},
				ylabel={$P$, fine in dollars},
				xtick={0,...,6},
				ytick={0,500,...,3500},
				]
				% use TeX as calculator:
			\end{axis}
		\end{tikzpicture}
\end{center}
\begin{itemize}
	\item \
	\item \
	\item \
\end{itemize}

\end{minipage}
\begin{minipage}{0.5\linewidth}
\begin{center}
	\captionof{figure}{City of Yonkers}
	\label{fig:cy5}
			\begin{tikzpicture}
			\begin{axis}[
			height=2in,
			width=2.75in,
			grid=major,
				xmin=0,xmax=6,
				ymin=0,ymax=3600,
				xlabel={$t$},
				ylabel={$P$},
				xlabel style={at={(ticklabel cs:0.5)},anchor=near ticklabel},
      	ylabel style={at={(ticklabel cs:0.5)},rotate=90,anchor=near ticklabel},
				xlabel={$t$, time in days},
				ylabel={$P$, fine in dollars},
				xtick={0,...,6},
				ytick={0,400,...,3600},
				]
			\end{axis}
		\end{tikzpicture}
\end{center}
\begin{itemize}
	\item \
	\item \
	\item \
\end{itemize}

\end{minipage}

\end{ex}

\vspace{11pt}

\begin{gw}
On what day will the city of Yonkers' fine reach over \$1,000,000? \\[2in]
\end{gw}

\begin{gw}
How much will the city of Yonkers be fined on day 30? What will each of the Councilmen's fines be on that day?
\end{gw}


\end{document}
